\section{Distributed Processing}

In order to lighten the processing load of the main host CPU we are interested in distributing real-time audio processing jobs to remote CPUs connected by gigbit ethernet. On the host CPU an audio plugin, loaded into a standard audio production application, functions as the master node, distributing jobs to networks slave nodes on the remote CPUs. To the host audio production software, the distribution of jobs should be completely transparent. The master node recieves audio and control data from the host application and returns the results just like any other audio plugin.

In contrast to other distributed processing environments where large data sets are parceled out to worker nodes, the plugin master is only given access to the audio data in small buffers as it is needed. The plugin then has a very small amount of time to process the data and pass it back to the host application. This puts some limits on how processing jobs can be distributed.

There are various degrees to which processing jobs can be distributed.


\begin{figure}[h]
    \centering
    \includegraphics[width=\textwidth]{assets/distribution_1to1.pdf}
    \caption{Each Plugin Distributes to One Node}
    \label{fig:one_to_one}
\end{figure}

