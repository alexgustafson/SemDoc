\section{Ausgangslage}

Im Digitalen Audio Bereich übernimmt die CPU immer mehr Aufgaben, die früher durch spezialisierte Hardware gelöst wurden. Dies beinhaltet nicht nur das Aufnehmen und Abmischen von Tonspuren, sondern auch die Klangerzeugung selbst. In den letzten Jahren stiegen die Erwartungen der User bedeutend schneller als die Entwicklung der CPU Performance. Von aufwendigen physikalischen Emulationen über Pianos und Gitarren bis hin zu virtuellen Analog Synthesizern, alles wird mittlerweile erwartet, was schnell zur Folge haben kann, das die CPU überlastet wird.

Ähnlich wie GPU-Karten, die Grafiken und 3D-Anwendungen beschleunigen, können Audio-DSP-Karten digitales Audio in Echtzeit verarbeiten, und dabei den Last auf der CPU Host-Computer verringern. Audio DSP-Karten werden in der Regel  über PCI, FireWire oder Thunderbolt an das Rechner verbunden. Die meisten Anbieter von DSP-Karten bieten die Möglichkeit, mehrere Karten parallel zu verbinden, um die Verarbeitungskapazität zu erhöhen.

Im Gegensatz zu GPU-Prozessoren hat sich jedoch kein offener Standard entwickelt, wie OpenGL oder OpenCL. 3D-Grafikanwendungen Gewinnen enorm von der Interoperabilität, die OpenGL bietet. Keine solche Vorteil existiert für digitale Audio-Anwendungen. Auch im Gegensatz zu OpenGL-Anwendungen, Audio-Software, die für einen Audio-DSP-Karte entwickelt wird, kann nicht auf der CPU-Host ausgeführt werden oder auf andere Audio-DSP-Karten. Dies führt zu "Vendor Lock-in". Der Verbraucher, der in einem Audio-DSP-Karte und Software investiert, muss beim gleichen Hersteller bleiben. Wenn ein anderer Anbieter von DSP-Hardware ein überlegenes Produkt herstellt, wird ein kaum den Plattform wechseln, wenn eine Investition bereits gemacht wurde.

Vor 10 Jahren war dies ein akzeptabler Kompromiss, weil DSP-Prozessoren über PCIe verbunden eine deutliche Leistungssteigerung ermöglichten. Heute könnten jedoch, ARM-basierte kostengünstige CPUs über Standard-Gigabit-Ethernet Verbunden eine wettbewerbsfähige Alternative bieten.

\section{Ziel der Arbeit}

Das Ziel dieser Studienarbeit ist es, die Möglichkeiten für die Verteilung von Audio Verarbeitungsaufgaben für einen Software-basierten Musik Synthesizer über ein Netzwerk von SBC-Geräten (Single Board Computer) zu untersuchen. Einschränkungen in der Polyphonie oder der Prozessorleistung sollen durch einfaches Hinzufügen eines neuen Geräts im Netzwerk gelöst werden. Ein besonderes Augenmerk soll auf günstige Raspberry-Pi und ähnliche SBC Geräte gelegt werden, vor allem unter Berücksichtigung von Preis und Leistung.
Es existieren viele Standards, damit Audio Applikationen miteinander kommunizieren können. MIDI und OSC definieren Protokolle zur Steuerung von Audio- Geräten und Software. VST, AU und LW2 sind Standard Schnittstellen für Plugins, die Echtzeitaudioverarbeitung bieten. AVB ( IEEE 1722 ), Jacktrip und Dante sind Standards bzw. Software für die Übertragung von hochwertigem Audio über Ethernet mit minimaler Latenz. Diese Studienarbeit wird einige dieser Standards untersuchen, um zu eruieren, was notwendig ist, um einen skalierbaren und verteilbaren Musik-Synthesizer zu entwickeln, der mit anderen professionellen Audio Software kompatibel ist.
Eine Proof-of-Concept Version der verteilbaren Synthesizer Software wird entwickelt, welche es einem Benutzer erlauben wird, ein Musikstück in Echtzeit zu spielen.

\section{Aufgabenstellung}

\begin{itemize}

\item Anforderungsanalyse mit Prioritätsbewertung

\item Vergleich von mehreren CPUs und Embedded Systems ( Banana Pi, Adapteva, Odroid) hinsichtlich ihrer Nutzbarkeit als Echtzeit Audioverarbeitungsmodule. Mit dem System, das die Anforderungen am besten erfüllt, wird die Implementierung gemacht.

\item Entwicklung der Audioverarbeitungssoftware in C ++.

\item Entwicklung eines VST-Plugins in C++, das als Schnittstelle zwischen gängigen Audio-Software und den Audioverarbeitungsmodule (pkt 3) dient.

\item Analyse der Implementierung, um die Nützlichkeit und Skalierbarkeit zu bewerten. Es ergeben sich dadurch verschiedene Fragestellungen wie z.B. folgende: Kann die Leistung und Polyphonie durch Hinzufügen weiterer Module erhöht werden, oder wird der Kommunikations-Overhead schließlich zu gross?

\end{itemize}


