\section{Ausgangslage}

20 years ago the CPU was just one component of a typical music studio. It was generally used to control and synchronize other equipement such as mixing boards, multitrack recorders, synthsizers and effects processors. Today all of the other equipement exists as software, running in realtime on a CPU host. A typical music studio today is comprised of a CPU, multiple analog to digital inputs and outputs, and some DSP equiped audio processing cards.

Simliar to GPU Cards which can accelerate graphics and visualization applications, audio DSP cards can process multiple streams of high qualtiy digital audio, eliviating the load on the CPU Host Computer. Audio DSP cards typically connect to the CPU via PCI, Firewire, or Thunderbolt. Most vendors of DSP cards offer the possibility to connect several cards in parallel to increase the processing capacity.

Unlike GPU processors however, no open standard has evolved to distribute the load across multiple co processors in the way it has for openGL or openCL. 3D graphics applications profit enormously from the interoperability that openGL offers. No such benefit is available for digital audio applications. Also, unlike OpenGL applications, audio software that is developed to run on an audio DSP card cannot be run on the CPU host. This results in vendor lock-in. The consumer that invests in an audio DSP card and software, must continue to buy from the same vendor in order to build on the the initial investment. If another vendor of DSP hardware creates a superior product, a consumer is unlikely to switch platforms if a significant investment has already been made.

10 years ago this was an acceptable compromise because DSP processors connected via PCIe could provide a significant performance increase. Today however, arm based inexpensive CPUs connected via standard gigabit ethernet could offer a competative alternatvie. 

\section{Ziel der Arbeit}

The purpose of this semester project is to design a software based music synthesiser that will run on a network of low cost banana pi devices. Limitations in polyphony will be alleviated by adding a new device to the network. In order to be compatible with existing music recording and composition applications the software will include a VST Plugin that allows music software to send MIDI commands to, and receive audio data from the software. All data communication between the VST Plugin and the Banana PI audio generation software will be handled via ethernet. The VST Plugin will send control data information such as pitch, volume, length, and other expression data. The Banana PI will stream back the generated audio data, as well as necessary metadata so the VST plugin can properly collect and prepare the audio data for the host software.

\section{Aufgabenstellung}

\begin{itemize}

\item Anforderungsanalyse mit Prioritätsbewertung

\item Vergleich von mehreren CPUs und Embedded Systems ( Banana Pi, Adapteva, Odroid) hinsichtlich ihrer Nutzbarkeit als Echtzeit Audioverarbeitungsmodule. Mit dem System, das die Anforderungen am besten erfüllt, wird die Implementierung gemacht.

\item Entwicklung der Audioverarbeitungssoftware in C ++.

\item Entwicklung eines VST-Plugins in C++, das als Schnittstelle zwischen gängigen Audio-Software und den Audioverarbeitungsmodule (pkt 3) dient.

\item Analyse der Implementierung, um die Nützlichkeit und Skalierbarkeit zu bewerten. Es ergeben sich dadurch verschiedene Fragestellungen wie z.B. folgende: Kann die Leistung und Polyphonie durch Hinzufügen weiterer Module erhöht werden, oder wird der Kommunikations-Overhead schließlich zu gross?

\end{itemize}


