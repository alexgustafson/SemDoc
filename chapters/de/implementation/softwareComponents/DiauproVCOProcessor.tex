\subsection{DiauproVCOProcessor}

Der DiauproVCOProzessor erbt vom DiauproProzessor und implementiert den Oszillatorblock eines virtuellen Synthesizers. Er erzeugt einen Klang in der Tonhöhe einer gespielten MIDI-Note. Die einzigen Funktionen, die vom Diaupro-Prozessor überschrieben werden müssen, sind ”localProcess”, ”getState”, ”getStateSize” und ”getServiceTag”.

"localProcess" wird wie folgt implementiert:
\begin{lstlisting}[numbers=left]
void DiauproVCOProcessor::localProcess(AudioSampleBuffer &buffer,
                                       MidiBuffer &midiMessages,
                                       void* state)
{
  processState = *(vco_state*)state;
  int sampleNr;
  int nextEventCount = -1;
  MidiBuffer::Iterator midiEventIterator(midiMessages);
  MidiMessage nextEvent;
  bool hasEvent;

  for(sampleNr = 0; sampleNr < buffer.getNumSamples(); sampleNr++)
  {
    if(nextMidiEventCount < sampleNr)
    {
      hasEvent = midiEventIterator.getNextEvent(nextEvent,nextEventCount )
    }
    if(hasEvent && nextEventCount == sampleNr)
    {
      if(nextEvent.isNoteOn())
      {
        processState.voice_count++;
        processState.frequency = MidiMessage::getMidiNoteInHertz(nextEvent.getNoteNumber());
        double cyclesPerSample = processState.frequency / getSampleRate();
        processState.step = cyclesPerSample * 2.0 * double_Pi;

      } else{
        processState.voice_count--;
      }
    }
    if(processState.voice_count > 0)
    {
      const float currentSample = (float) (sin (processState.phase) * processState.level);
      processState.phase += processState.step;
      for(int i = 0; i < buffer.getNumChannels(); i++)
      {
        float oldSample = buffer.getSample(i, sampleNr);
        buffer.setSample(i, sampleNr, (currentSample + oldSample)*0.5);
      }
    }
  }
}
\end{lstlisting}

\noindent
Wenn die Funktion aufgerufen wird, wird der aktuelle Audio-Puffer übergeben, ein Puffer von MIDI-Events für den entsprechenden Zeitrahmen und ein Zeiger auf einen Block von Daten, die zur Zustandsspeicherung zwischen den Aufrufen verwendet werden? ??

Wird diese Funktion innerhalb eines Rechenknotens ausgeführt, dann werden Audio-, MIDI- und Zustandsdaten aus einer DiauproMessage, die als UDP-Paket vom Master-Audio-Plug-in gesendet wurde, entnommen.

Im obigen Beispiel wird eine Sinuswelle generiert, indem der Sinus-Wert für jedes Element im Audio-Puffer berechnet wird.