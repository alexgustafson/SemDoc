\subsection{ZeroconfManager}

Zeroconf (kurz für Zero Configuration Netzwerking, auch als Bonjour auf OSX bekannt) ist einer Technologie, die es einen Softwaredienstleistung erlaubt ihre Verfügbarkeit und Standort über einen Netzwerk bekannt zu machen. Drucker und Multimedia-Geräte verwenden Zeroconf z.B., damit sie für andere Netzwerkteilnehmer leicht zu finden und nutzen sind.

Bonjour auf OSX und den compatibile Funktionen von Avahi unter Linux definieren eine Rückruff-basiertes API zu der Bonjour oder Avahi Daemon-Prozess. Die API ist in C. Die ZeroConfManager-Klasse kapselt die API in einer objektorientierten Weise. Andere Klassen, welche die Zeroconfigdienste ansprechen müssen können von der ZeroConfListenere-Klasse erben, und können sich bei der ZeroConfManager-Klasse als Zuhörer registrieren.

Wird ein neuer Software-Dienst im Netzwerk registriert, oder verschwindet ein Bestehender aus dem Netzwerk, werden die entsprechenden Zuhörer einer Liste aller aktiven Services mitgeteilt. Wenn sie noch mit einem Software-Dienst, der nicht mehr verfügbar ist, verbunden sind, muss es sich von diesem Dienst trennen.

Um einen Dienst an eine bestimmte IP-Adresse und Port-Nummer aufzulösen, müssen mehrere asynchrone Aufrufe an der Bonjour-Daemon vorgenommen werden. Zwischen den Aufrufen müssen die Ergebnisse gespeichert werden und den Zustand aktuallisiert. Die ZeroConfManager verbirgt diese Komplexität von den Zuhörern und benachrichtigt sie nur, wenn alle Informationen vorhanden sind.