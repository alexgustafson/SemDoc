\subsection{DiauproMessage}

Die DiauproMessage-Klasse verwaltet die Serialisierung von Daten in Datagramm-Pakete bzw. deren Deserialisierung. Die Datagramme selbst bestehen aus einem Header mit festgelegter Länge und Nutzdaten mit variabler Länge, welche die Audio, MIDI- und Zustandsdaten enthalten.\\
\\
Der Header wird wie folgt definiert:

\begin{lstlisting}
struct diapro_header {
    uint16 sequenceNumber;
    uint16 numSamples;
    uint16 numChannels;
    double sampleRate;
    uint16 audioDataSize;
    uint16 midiDataSize;
    uint16 stateDataSize;
    double cpuUsage;
    double totalTime;
    double processTime;
    uint32 tagNr;
};
\end{lstlisting}

Die DiaproMessage-Klasse bemüht sich, möglichst bestehenden, zugewiesenen Speicher zu verwenden. DiauproMessage-Instanzen sollten nicht dem Thread zugewiesen werden, der die Audio-Verarbeitungsroutinen aufruft. Sie sollte vielmehr vorab mit genügend Speicher belegt werden, um das größtmögliche UPD-Datagramm  (64 KB) zu speichern. Diese Instanz kann dann für jeden Aufruf-Zyklus wiederverwendet werden, so dass während einer kritischen Prozessphase nicht erneut Speicher zugewiesen werden muss.
