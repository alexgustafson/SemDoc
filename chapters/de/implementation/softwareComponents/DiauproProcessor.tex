\subsection{Diaupro-Prozessor}

Die DiauproProzessor-Klasse ist die Basisklasse, von der andere Prozessoren erben sollen. Die DiauproProzessor-Klasse erbt ihrerseits  von der JUCE Audioprozessor-Klasse, die alle Funktionen der verschiedenen Audio-Plug-in-Formate einkapselt. Klassen, die von Audioprozessoren erben, können direkt in ein VST-Plug-in implementiert werden.

Eine Klasse, die vom Diaupro-Prozessor erbt, muss die Funktionen „localProcess” und „getServiceTag” überschreiben. Die „localProcess”- Funktion führt die Audio-Datenverarbeitung aus. Die „getServiceTag”-Funktion liefert eine Zeichenkette, die verwendet wird, um ihre Dienste im Netzwerk anzuzeigen oder danach zu suchen.

Klassen, die vom Diaupro-Prozessor erben, können instanziiert und eingerichtet werden, sodass sie entweder im Audio-Plug-in oder im Rechenknoten laufen. Im Plug-in-Code ausgeführt, suchen sie einen entsprechenden Rechenknoten im Netzwerk. Falls nichts Passendes gefunden wird, wird die „localProcess”- Funktion lokal verwendet. Finden sie einen entsprechenden vernetzten Service, werden alle ankommenden Audio- und MIDI-Daten an diesen weitergeleitet.

Wenn in einem Rechenknoten ausgeführt, melden sich Klassen, die vom Diaupro-Prozessor erben, im Netzwerk an und warten auf ankommende Verarbeitungsanforderungen.

Beispiele für Klassen, die vom Diaupro-Prozessor erben, sind die Diaupro-VCO- und  Diaupro-VCA-Prozessoren. Sie werden weiter unten detailliert beschrieben.
