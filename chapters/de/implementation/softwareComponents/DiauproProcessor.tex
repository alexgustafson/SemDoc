\subsection{DiauproProcessor}

Die DiauproProcessor-Klasse ist die Basisklasse, von der anderen Prozessoren erben sollen. Die DiauproProcessor-Klasse erbt wiederum von der JUCE Audioprocessor-Klasse, die alle Funktionen der verschiedenen Audio-Plugin-Formate kapselt. Klassen, die von Audioprocessor erben können direkt in ein VST-Plugin gewickelt (wrapped) werden.

Eine Klasse, die von DiauproProcessor erbt muss die Funktionen "localProcess" und "getServiceTag" überladen. Die "localProcess" Funktion führt die Audio-Verarbeitung aus. "getServiceTag" gibt einer Zeichenfolge zurück, die verwendet wird, um seine Dienste im Netzwerk bekannt zu machen oder danach zu suchen.

Klassen, die von DiauproProcessor erben können instanziert und eingerichtet werden so, dass sie entweder in der Audio Plug-in oder in der Rechenknoten laufen. Im Plug-in ausführt wird es ein entsprechendes Rechenknoten im Netzwerk suchen. Falls nichts passendes gefunden wird, wird das "localProcess" Funktion lokal verwendet. Es sie finden eine vernetzte Service für alle ankommenden Audio- und MIDI-Daten werden auf diesen Dienst weitergeleitet.

Wenn in einem Rechenknoten ausgeführt, werden Klassen, die von DiauproProcessor erben sich im Netzwerk anmelden und warten, um ankommende Verarbeitungsanforderungen.

Beispiele für Klassen, die von DiauproProcessor erben sind DiauproVCOProcessor und DiauproVCAProcessor. Sie werden unten in Detail beschrieben.