\section{Software-Komponenten}

Die Anwendung ist in Komponenten (Klassen) aufgebaut, welche jeweils die Verantwortung für einen Teilbereich steuern. Jedes Modul wurde in einer „Pseudo-Test-driven” Methodik entwickelt. Die Tests wurden nicht unbedingt entwickelt, bevor der entsprechende Code geschrieben wurde, aber auf jeden Fall kurz danach. Dies führte zu einem stabilen Code, der in jeder Code-Iteration geprüft werden konnte, isoliert von anderen Komponenten.

Die Interaktion mit einer Klasse wird in einer dazugehörigen „Zuhörer” Klasse definiert mit spezifischen Rückrufmethoden, die vom Klienten zu implementieren sind. Eine Client-Klasse kann dann von der Listener-Klasse erben und sie überschreiben.