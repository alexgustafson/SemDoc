\section{Background}

Vor 20 Jahren musste man ins Tonstudio um einen Musik- oder Audioaufnahme zu bearbeiten. Dort befanden sich großer Racks voller Geräte für verscheidene Signalverarbeitungsaufgaben. Zum Beispiel, Komprssoren und Limiters um den Dynamikbereich zu verarbeiten, oder Hall und Echo geräte um einen Tonspur mehr Atmosphäre zu vergeben. Am Mischpult konnte man Lautstärke und Frequenzgang mehrere Tonspuren anpassen.

Heute werden alle diese Aufgaben von Software-Plug-ins auf der CPU berechnet.

1996 hatte Steinberg GmbH, die Entwickler von Cubase, einem populären Audio-Produktionssoftware (oder DAW, Digital Audio Workstation), die VST-Schnittstellenspezifikation und SDK veröffentlicht. Das VST-Plugin-Standard war Besonderes, weil es Echtzeit-Verarbeitung von Audiodaten in der CPU ermöglichte. Dazu hat konnten andere Entwickler Plugins entwickeln die innerhalb Cubase ausgeführt werden konnten. Das VST-Plugin-Standard hatte schnell Akzeptanz in der Branche gefunden und wurde sogar von konkurrierenden DAWs implementiert. Obwohl alternative Standards heute existieren, ist VST immer noch die am weitesten verbreitete Crossplatform-Standard.

Die Zahl der Echtzeit-Plug-ins, die auf einer CPU laufen konnte, wurde durch mehrere Faktoren wie, Festplattenzugriffsgeschwindigkeiten, Bus-Geschwindigkeiten, viel RAM und OS-Scheduler zum Beispiel beschränkt \cite{brandt1998low}. Benutzer damals erwarteten nicht mehr als 10 Plugins gleichzeitig ausführen zu können. Schon der Wiedergabe mehrerer Spuren ohne Plug-ins konnte einen Rechner ins Stottern bringen.

Heute ist es jedoch möglich, hunderte Tonspuren und Hunderte Plugins in Echtzeit wiederzugeben. Während die Leistungsgrenze angestiegen ist, stiegen auch die Erwartungen der Benutzer. Die Algorithmen heutiger Plugins sind sehr viel komplexer, als diejenigen von 1996. Es werden Akustische Systeme detailgenau Modeliert und die Schaltkreise alte 70er Jahre Synthesisers digital emuliert. Auch wenn die CPU-Leistung sich deutlich erhöht hat, ist es noch leicht, die Grenzen zu erreichen.

Mehrere DSP-basierten Systemen existieren, die, ähnlich wie GPU-Beschleunigungskarten,  die CPU Belastung lindern. Audio-Verarbeitungsaufgaben werden an externe Hardware via PCIe oder Thunderbolt-Schnittstellen übertragen. Jedoch sind diese DSP-basierten Systeme proprietäre und teuer. Die Entwicklung von Software für einen DSP-Chip ist auch einiges komplexer als für einen CPU.