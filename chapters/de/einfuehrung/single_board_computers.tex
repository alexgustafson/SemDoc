\section{Single Board Computers}

Die Popularität der Raspberry Pi hat eine ganze Industrie rund um Single-Board-Computer (SBC) generiert. Basierend auf Hardware welches in Mobiltelefonen zu finden ist, sind diese kleinen Low-Power-Geräte sehr beliebt, weil sie preiswert und einfach zu bedienen sind. Der größte Vorteil von SBCs im Vergleich zu anderen eingebetteten Geräten ist, dass sie mit Android- und Linux-Betriebssysteme ausgeführt werden, so dass sie mit den gleichen Tools auf Desktop-Computern zur Verfügung programmiert werden.

Aktuelle High-End-SBCs werden sogar mit Gigabit-Ethernet ausgestatet und Quad-Core-CPUs mit Geschwindigkeiten weit über 1 GHz getaktet. Vergleicht man diese Systeme mit den 450 MHz G3 PPC-Systeme auf dem die ersten VST Software Plug-ins berechnet wurden, sollten wir erwarten können, dass die neueren High-End-SBCs exzellente Audio Coprozessoren sein sollen.

Zwei SBCs sind es wert, in Betracht gezogen zu werden, da sie möglicherweise noch bessere Performance als Audio Coprozessoren beiten. Das Parallella Board hat einen 16 Core Epiphany Coprozessor welches verwendet werden kann, um die Audioverarbeitung parallel auszuführen. Standard Frameworks wie OpenCL, MPI oder OpenMP können verwendet werden, um die Epiphany-Cores zu benützen. Der ODROID-XU4 SBC enthält eine Mali-T628 GPU-Coprozessor, der auch OpenCL kompatibel ist. Beide sind für weniger als \$ 100 erhältlich.

Der Programmierung von Audioverarbeitungsalgorithmen als OpenCL Kernels könnte einiges komplexer sein als in C ++, aber OpenCL bietet herstellerunabhängige Zugriff auf GPGPU-Computing und hat den zusätzlichen Vorteil, dass es auch auf einem CPU ohne GPU-Beschleunigung verwendet werden kann\cite{vst-gpu}.

Die Untersuchung dieser und anderer OpenCL fähiger SBCs könnte eine interessantes Folgeprojekt sein.