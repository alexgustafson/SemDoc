\section{Echzeit Audio Plug-ins}

Das Produzieren von Audio und Musik wird in der Regel mit Hilfe eines Digital Audio Workstations (DAW) Software gemacht. MIDI-Events und Audio-Aufnahmen werden als Spuren arrangiert, gemisht, editiert und verarbeitet. Um Änderungen rückgängig zu machen, werden Bearbeitungen in einer Nicht-destruktives Verfahren gemacht, dynamisch in Echtzeit während der Wiedergabe berechnet. Die ursprünglichen Audiodaten beleiben immer im original Zustand erhalten. Der Benutzer kann die Parameter eines Effektes oder Prozess in Echtzeit ändern, und mit verschiedenen Einstellungen experimentieren, ohne zu fürchten, dass die ursprüngliche Audioaufzeichnungen geändert werden.

Ein DAW-Anwendung hat in der Regel mehrere Echtzeit-Effekte eingebaut, welche ein Benutzer auf Audiospuren anwenden kann. Zusätzlich zu den eingebauten Effekte sind alle professionellen DAW-Anwendungen auch in der Lage, Dritthersteller-Plugins zu laden. In Abhängigkeit von der Plattform und Anbieter werden einer oder mehreren Plugin-Standards implmeneted. Die am weitesten verbreitete Standard ist Steinbergs VST-Standard.

Alle Standards funktionieren in einer ähnliche Weise. Der Host DAW-Anwendung übergibt in regelmäßigen Abständen das Plug-in über einen Rückruffunktion zu-verarbeitende Audiodaten. Das Plug-in muss innerhalb einer definierten Zeitrahmen die Verarbeitung abgeschlossen haben und wartet dann auf dar nächste Abruff.

Audio-Plug-ins können auch einen Grafisches Oberfläche zu verfügung stellen über die ein Benutzer Parameter ändern und speichern kann. Dies könnte die Grenzfrequenz eines Tiefpassfilters oder der Verzögerungszeit eines Hall-Effekt sein, zum Beispiel.

Aus der Sicht des Programmierers sind Plug-ins dynamisch ladbare Bibliotheken, die ein spezifiziertes API implementieren. Der Host-DAW-Anwendung dann sie zur Laufzeit laden und Audiodaten durch sie verarbeiten lassen\cite{realtime-architectures}. Auf der Windows-Plattform werden VST-Plugins als \(Dynamic\) \(Link\) \(Libraries\) (DLL) kompiliert, auf Mac OSX sind sie Mach-O-Bundles. Die native Apple AudioUnit Plug-ins sind auch als Mach-O Bundel zusammengestellt, sie haben fast identische Funktionalität. Andere alternative Plugin Formate sind Avids RTAS und AAX-Plug-Formate, Microsofts DirectX-Architektur oder LADSPA, DSSI und GW2 auf Linux.

