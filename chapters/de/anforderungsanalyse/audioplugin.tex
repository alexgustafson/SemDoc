\section{Audio Plugin}

Aus der Sich des Host DAW-Anwendungs muss das Plug-in wie ein ganz normales VST-Plugin aussehen. Das Verteilen der Aufgaben auf vernetzte Rechenknoten muss transparent funktionieren. Das Plug-in wird nicht in einer isolierten Umgebung ausgeführt. Es teilt die Rechenresources mit der Host DAW, so wie eine beliebige Anzahl anderen Plugins. Daher sollte sorgfälltig die Belastung auf der CPU auf das Nötigste minimiert werden.

Audio Plug-ins haben typischerweise einen Zustand, der Parametereinstellungen enthält die vom Audio-Host oder vom Benutzer gesteuert werden können. Der aktuelle Zustand muss den vernetzten SBC Modulen auch mit geteilt werden. Dies ist notwendig, damit ein Audio Plug-in, der Zustand einer fortlaufender Prozess, nahtlos einer vernetzte Rechenknoten überreichen kann. Es könnte auch verwendet werden, um eine vernetzte Rechenknoten durch Umschalten des aktiven Zustandes die Verarbeitung für mehrere Audio Plug-ins zu übernehmen.\\
\\
\noindent
Das Audio-Plugin stellt die folgenden Anforderungen:

\begin{itemize}

\item lauffähig als Echtzeit VST Audio Plug-in in einer Standard DAW Anwendung
\item lokalisieren und verbinden mit einer passende Rechenknoten im Netzwerk
\item überreichen von MIDI- und Audio-Daten vom Host DAW-Anwendung, zu den vernetzten Rechenknoten
\item empfängt Audiodaten von den Rechenknoten und übergibt diese zurück an der DAW-Anwendung
\item ist keinen entsprechenden Knoten im Netzwerk vorhanden, muss die Audioverarbeitung Lokal statfinden
\item die Audio Plug-in muss sienen aktuellen Zustand an den Rechenknoten übermitteln können

\end{itemize}