\section{Remote Processing Node}

Die Rechenknoten sind Anwendungen, die auf den vernetzten SBC-Geräten laufen. Je nach Art der implementierte Parallelisierung in der Audio Plug-in wird das Rechenknoten die gesamte Verarbeitungsaufgabe oder einfach nur ein Teil der Aufgabe bekommen. Für dieses Projekt werden die Rechenknoten als Reihe von Prozessoren in eine übergeordnete Anwendung umgesetzt werden. Die übergeordnete Anwendung wird eine Socket-Listener implementieren, welche den zugeorneten Rechenknoten aufrufft sobald ein Audio Verarbeitungsaufgabe eintrifft. Die Verfügbarkeit, Art und Lage der Rechennoten und das entsprechende Socket wird über das Netzwerk via Bonjour / Zeroconf übertragen.

Die Rechenknoten sollten stateless sein. Jeder Zyklus des Audio-Verarbeitungsalgorithmus sollte nur den Zustandsdaten des entsprechenden Pakets berücksichtigen müssen. Werden die Zustandsdaten durch den Verarbeitungsalgorithmus geändert müssen sie an das Audio Plug-in zurück gesendet. Damit soll sichergestellt werden, dass ein Rechenknoten zu jeder Zeit in ein laufendes Session einspringen ohne etwas über vorherrige Events wissen zu müssen. Es hat auch den zusätzlichen Vorteil, dass ein Verarbeitungsknoten in der Lage wäre, Aufträge für mehrere Instanzen eines bestimmten Plug-ins verarbeiten zu können.\\
\\
\noindent
Die Rechenknoten müssen die folgenden Anforderungen erfühlen:

\begin{itemize}

\item gibt seine Verfügbarkeit und Standort im Netzwerk via Bonjour / Zeroconf bekannt
\item akzeptiert Steuerdaten von der Audio Plug-in
\item verarbeitet eingehende Audiodaten und MIDI-Daten von der Audio Plug-in.
\item schickt die verarbeitende Daten sofort an das Audio Plug-in zurück.

\end{itemize}