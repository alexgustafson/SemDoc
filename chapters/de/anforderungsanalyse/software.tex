\section{Software Requirements}

In Echtzeit-Audio-Anwendungen das Einhalten zeitlicher Vorgaben ist kritisch. Dies mag selbstverständlich klingen, aber für einen Programmierer bedeutet es den Verzicht auf viele Komforten der modernen Programmierung, besonders das Arbeiten mit High-Level Programmier-Sprachen wie Java oder Python. Echzeit-Audio ist ein "Hard Realtime" Aufgabe und dafür eignen sich nur Low-Level Sprachen wie C oder C++. Auch die meisten Audio-Anwendungsschnittstellen und Bibliotheken wie die VST SDK sind für C/C ++.

Professionelle Audio-Anwendungen laufen in der Regel auf Mac OSX oder Windows-Betriebssysteme, daher muss der Audio Plug-in auf diesen Systemen Kompilierbar sein. Die Rechenknoten, welche auf den SBC-Geräte laufen, müssen für Linux Kompilierbar sein. Doch beide Anwendungen sollten ein gross teil ihre Quellcode teilen können, da ihre Aufgaben sich grossteils überschneiden.

Es gibt viele C ++ Bibliotheken und Frameworks, welche Cross-Plattform-Kompatibilität ermöglichen und nebenbei  auch den Programmierer Zugriff auf High-Level Konstrukte wie  automatischen Speicherbereinigung mittels intelligente Zeiger und Referenzzählung, die das Programmieren in C++ einfacher machen.

\subsection{Evaluierte C++ Frameworks}

Boost ist die beliebteste plattformübergreifende C ++ Framework. Viele Boost Funktionalitäten wurden sogar in die C ++ 11 Standard-Bibliothek hinzugefügt. Andere Frameworks wie Cinder und Open Frameworks bieten viele High-Level Funktionen, um schnell interaktive medien-reiche Anwendungen zu programmieren. Zwei Bibliotheken die hervorzuheben sind, JUCE und WDL, beiten besonders für Audio Plug-ins und DAW-Anwendungen zugeschnittene Funktionen und Klassen. Von diesen beiden hat Juce eine viel größere Benutzergemeinschaft (einschließlich Anbieter von DSP-basierten Audio-Coprozessoren). \\
\\
\noindent
Software Framework Kriteria:

\begin{itemize}

\item Plattformübergreifend für OSX, Linux und Windows
\item Bietet Hohe Konstrukte wie intelligente Zeiger
\item Unterstützung für plattformübergreifende Audiointegration
\item Sollte gut dokumentiert sein und eine aktive Benutzergemeinschaft haben
\item Bietet plattformübergreifende Netzwerkzugriff

\end{itemize}


\begin{table}[H]
\begin{center}
\begin{tabular}{ |p{3cm}||p{1.5cm}|p{1.5cm}|p{1.5cm}|p{1.5cm}|p{1.5cm}|  }
 \hline
 Framework & High Level Utilities & Audio Utilities & Network Utilities & VST Utilities & Community\\
 \hline
 JUCE            & ja  & ja  & ja\tablefootnote{basic socket management classes} & ja  & gross \\
 WDL             & ja  & ja  & nein & ja\tablefootnote{enabled using one of the additional iplug libraries} & klein \\
 Open Frameworks & ja  & ja  & ja\tablefootnote{the ofxNetwork addon allow simple management of TCP or UDP sockets}&
 nein& gross \\
 Boost           & ja  & nein& ja   & nein & gross \\
 Cinder          & ja  & ja  & ja   & nein & klein \\
 LibSourcey      & nein& nein & ja & nein & nein \\
 Qt              & ja  & nein & ja & nein & gross \\
 \hline
\end{tabular}
\end{center}
\end{table}

Auf Grundlage der Kriterien Vergleich und früheren Erfahrungen bei andere Projekten wurde für diese Projekte JUCE für die Implementierung ausgewählt.