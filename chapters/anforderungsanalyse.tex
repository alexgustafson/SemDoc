\section{General Application Requirements}

The Application has two components, the audio plugin hosted on the main CPU machine, and the processing node which runs on a networked SoC device. The audio plugin forwards midi control and audio data to the processing nodes. The nodes stream the processed audio data back to the audio plugin, which in turn streams it back to the host audio application. The total round-trip time, including processing, should not exceed 10ms. This is the maxium allowed latency for live sound applications. \cite{AES67-2013}

The applications must be self contained and work without the user needing to install any system libraries, frameworks or servers.\footnote{The only exception might be ZeroConf/Bonjour on Linux or Windows. See Appendix}

\subsection{Audio Plugin}

The audio plugin has the following requirements:

\begin{itemize}

\item runnable as a realtime VST audio plugin in a standard audio application
\item locate and connect with one or processing nodes on the network
\item forward midi and audio data from the host audio application to the networked nodes
\item receives audio data from the networked nodes and streams this back to the host application

\end{itemize}

\subsection{Processing Node}

The processing node has the following requirements:
\begin{itemize}

\item broadcasts its availability and location on the network
\item accepts session initiated by the audio plugin
\item accepts controll data from the audio plugin
\item processes incoming audio data and midi data from the audio plugin
\item streams audio and midi data back to the audio plugin or to the next node in the processing chain

\end{itemize}

\subsection{Software Requirements}

In realtime audio applications timing is critical. This may sound obvious, but to a programmer it means giving up many of the comforts of modern programming made available working with high level interpreted languages such as java or python. Most audio application interfaces and SDKs such as the VST SDK require a knowledge of C and C++.

Professional audio applications generally run on Mac OSX or Windows Operating Systems, therefore the audio plugin must be compileable on these systems. The processing node will be run on SoC devices which typically run with a Linux based OS. Yet both applications should share much of thier codebase since they must do similar things and be compatible.

\subsection{Evaluated C++ Frameworks}

There are many C++ libraries and frameworks that..\\

Software Framework Criteria:

\begin{itemize}

\item Crossplatform for OSX, Linux, and Windows
\item Offers high level constructs like smart pointers
\item Support for crossplatform audio integration
\item Should be well documented and have an active community
\item Support for crossplatform network streaming

\end{itemize}

Several C++ Frameworks were evaluated for this project. The criteria used to evaluate the frameworks are the ease of
integration, size of community and degree of acceptence. Does the framework cover the software requirements defined
above?

WDL : http://www.cockos.com/wdl/ ( +iplug library ) \\
Juce : http://www.juce.com \\
Open Frameworks : http://openframeworks.cc \\
Boost : http://www.boost.org \\
Cinder : http://libcinder.org \\
LibSourcey : http://sourcey.com/libsourcey/ \\
Qt : http://www.qt.io \\

\begin{table}
\begin{center}
\begin{tabular}{ |p{3cm}||p{1.5cm}|p{1.5cm}|p{1.5cm}|p{1.5cm}|p{1.5cm}|  }
 \hline
 Framework & High Level Utilities & Audio Utilities & Network Utilities & VST Utilities & Community\\
 \hline
 Juce            & ja  & ja  & ja\tablefootnote{basic networking utilities, not appyable to this project though} & ja  & gross \\
 WDL             & ja  & ja  & nein & ja\tablefootnote{enabled using one of the additional iplug libraries} & klein \\
 Open Frameworks & ja  & ja  & ja\tablefootnote{the ofxNetwork addon allow simple management of TCP or UDP sockets}&
 nein& gross \\
 Boost           & ja  & nein& ja   & nein & gross \\
 Cinder          & ja  & ja  & ja   & nein & klein \\
 LibSourcey      & nein& nein & ja & nein & nein \\
 Qt              & ja  & nein & ja & nein & gross \\
 \hline
\end{tabular}
\end{center}
\end{table}


