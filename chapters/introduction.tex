\section{Ausgangslage}

20 years ago the CPU was just one component of a typical music studio. It was generally used to control and synchronize other equipement such as mixing boards, multitrack recorders, synthsizers and
effects processors. Today all of the other equipement exists as software, running in realtime on a CPU host. A typical music studio today is comprised of a CPU, multiple analog to digital inputs and outputs, and some DSP equiped audio processing cards. 

Simliar to GPU Cards which can accelerate graphics and visualization applications, audio DSP cards can process multiple streams of high qualtiy digital audio, eliviating the load on the CPU Host Computer. Audio DSP cards typically connect to the CPU via PCI, Firewire, or Thunderbolt. Most vendors of DSP cards offer the possibility to connect several cards in parallel to increase the processing capacity.

Unlike GPU processors however, no standard similar to OpenGL has develped which allows software from one vendor to run on hardware from another. Also, unlike OpenGL applications, audio software that is developed to run on an audio DSP card cannot be run on the CPU host. This results in vendor lock-in,
the consumer that invests in an audio DSP card and software, must continue to buy from the same vendor in order to build on the the initial investment. If another vendor of DSP hardware creates a superior product, a consumer is unlikely to switch platforms if a significant investment has already been made.  

10 years ago this was an acceptable compromise because DSP processors connected via PCIe could provide a significant performance increase. Today however, arm based inexpensive CPUs connected via standard gigabit ethernet could offer a competative alternatvie. 

\section{Ziel der Arbeit}

