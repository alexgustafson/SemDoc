\subsection{ZeroconfManager}

Zeroconf (short for Zero Configuration Network and also know as Bonjour on OSX ) is a specification that allows services to broadcast their availability and location on a network. Printer and Multimedia devices use Zeroconf in order to allow networked computers to easily find and use their services.

Bonjour on OSX and the compatibile features implemented by Avahi on Linux define a callback based API that interfaces to a bonjour or avahi daemon running on the OS. The Bonjour API is in C. The ZeroConfManager class encapsulates communication to the bonjour or avahi deamon in an object oriented manner. Clients that want to interface with this class must extend and override the ZeroConfListener class, they register themselves with the class along with the service tag they are browsing for.

When new services are registered on the network, or removed from the network. The corresponding listeners are notified with a list of all active services. If they are currently connected with a service that is no longer available, it is the clienst responsibility to disconnect from that service.

In order to resolve a service to a specific IP address and port number serveral asyncronous calls must be made to the Bonjour daemon, saving and updating the state of each result between call. The ZeroConfManager hides this complexity from it's clients and only notifies them when all the information is final.




