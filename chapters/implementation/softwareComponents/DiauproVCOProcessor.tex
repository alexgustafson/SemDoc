\subsection{DiauproVCOProcessor}

The DiauproVCOProcessor extends the base DiauproProcessor and implements the oscillator block of a virtual synthsizer. The only functions from DiauproProcessor that need to be overrided are localProcess, getState, getStateSize, and getServiceTag.

localProcess is implemented as follows:
\begin{lstlisting}[numbers=left]
void DiauproVCOProcessor::localProcess(AudioSampleBuffer &buffer,
                                       MidiBuffer &midiMessages,
                                       void* state)
{
  processState = *(vco_state*)state;
  int sampleNr;
  int nextEventCount = -1;
  MidiBuffer::Iterator midiEventIterator(midiMessages);
  MidiMessage nextEvent;
  bool hasEvent;

  for(sampleNr = 0; sampleNr < buffer.getNumSamples(); sampleNr++)
  {
    if(nextMidiEventCount < sampleNr)
    {
      hasEvent = midiEventIterator.getNextEvent(nextEvent,nextEventCount )
    }
    if(hasEvent && nextEventCount == sampleNr)
    {
      if(nextEvent.isNoteOn())
      {
        processState.voice_count++;
        processState.frequency = MidiMessage::getMidiNoteInHertz(nextEvent.getNoteNumber());
        double cyclesPerSample = processState.frequency / getSampleRate();
        processState.step = cyclesPerSample * 2.0 * double_Pi;

      } else{
        processState.voice_count--;
      }
    }
    if(processState.voice_count > 0)
    {
      const float currentSample = (float) (sin (processState.phase) * processState.level);
      processState.phase += processState.step;
      for(int i = 0; i < buffer.getNumChannels(); i++)
      {
        float oldSample = buffer.getSample(i, sampleNr);
        buffer.setSample(i, sampleNr, (currentSample + oldSample)*0.5);
      }
    }
  }
}
\end{lstlisting}

\noindent
When the function is called it is passed a reference to the current audio sample buffer to be filled, a buffer of midi events for the corresponding time frame, and a pointer to a block of data that can be used to hold and state that needs to be persisted between calls.

If this function is called from a node processor then the audio, midi, and state data has been extracted from a DiauproMessage sent as a UDP packet from the master audio plugin.

In the example above a sin wave is create by calculating the sin value for each element in the audio sample buffer.