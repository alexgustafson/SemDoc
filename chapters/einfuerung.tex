\section{Hintergrund}

An audio engineer's typical job is to manage the balance of multiple tracks of audio signals. The dynamic range of a
signal can be compressed, in order to give quiter passages more presence. Loud peaks can be limited to balance the
overall loudness of a musical piece. Using equalizers an audio engineer can make enhance or supress specific
frequencies of a track to make it more present in a mix. Effects like reverb, echo, or chorus can be used to give a
track more space in a mix effecting the mood or ambience of a music piece. It is typcial that each track in a
recording session will be processed by a chain of several specific audio processors.

20 years ago the equipment responsible for these kinds of processes filled large racks. Today all of these tasks run
as plugins on the CPU.

In 1996 Steinberg GmbH, the developers of several very popular audio production applications released the VST
interface specification and SDK.\cite{VST-wikipedia} The VST plugin standard quickly had widespread industry
acceptance and was adopted by most developers of audio production applications. Although competeing standards exist
on Apple and Linux Operating Systems, VST is still the most widely adopted crossplatform standard.

The VST Plugin specification defines a set of classes and methods that must be implemented in order to process audio
passed from the host application.

TODO: describe here in more detail the basics of the VST plugin secification.. how an audio pluginn is implemented